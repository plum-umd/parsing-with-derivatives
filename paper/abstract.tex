We present a functional approach to parsing unrestricted context-free grammars
based on Brzozowski's derivative of regular expressions.
%
If we consider context-free grammars as recursive regular expressions,
Brzozowski's equational theory extends without modification to context-free
grammars (as well as parser combinators).
%
The technique handles left-recursive, right-recursive, infinitely
recursive and ambiguous grammars to produce a parse forest for any input.

The supporting actors in this story are three concepts familiar to functional
programmers---laziness, memoization and fixed points.
%
These techniques allow Brzozowski's original equations to be transliterated
into purely functional code; about 30 lines spread over three functions.
 

Yet, this almost impossibly brief implementation has a drawback: its
performance is sour---in both theory \emph{and} practice.
%
The culprit?
%
Each derivative can \emph{double} the size of a grammar, and with it, the cost
of the next derivative.


Fortunately, much of the new structure inflicted by each derivative is
detritus: it no longer contributes to the meaning of the grammar, and can be
removed.
%
To eliminate it, we once again exploit laziness, memoization and fixed-points
to transliterate an equational theory which prunes such debris.
%
We introduce two optimization under which drop parsing times dramatically. 

We equip the functional programmer with executable equational theories for
parsing with derivatives: these theories abbreviate 
understanding and implementation of parsing arbitrary context-free languages.



%%% Local Variables: 
%%% mode: latex
%%% TeX-master: "paper-sigplan.tex"
%%% End: 

